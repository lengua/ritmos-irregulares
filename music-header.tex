% filepath: c:\Users\lalen\Desktop\RITMOS IRREGULARES\TEXTOS\_FINALES\music-header.tex
% \documentclass[a4paper,14pt,twoside]{extbook}
% \documentclass[a4paper,14pt,twoside]{extbook}

% Idioma y codificación
% \usepackage[spanish,es-nodecimaldot]{babel}
\usepackage[utf8]{inputenc}
\usepackage[T1]{fontenc}

% Márgenes y geometría de página
% \usepackage[a4paper,margin=2.5cm]{geometry}
\usepackage[a4paper,inner=2cm,outer=2cm,top=2.5cm,bottom=2.5cm]{geometry}

% Paquetes matemáticos
\usepackage{amsmath, amssymb, amsfonts, amsxtra, mathdots, esint}

% Paquetes para símbolos musicales y gráficos
\usepackage{musicography}
\usepackage{musixtex}
\usepackage{wasysym}

% Tablas y gráficos
\usepackage{array}
\usepackage{booktabs}
\usepackage{graphicx}
\usepackage{float}

\usepackage{eso-pic}

\usepackage{nextpage}

% Hipervínculos y referencias
\usepackage{hyperref}
\hypersetup{
    colorlinks=true,
    linkcolor=brown,
    urlcolor=brown,
    citecolor=brown
}

% Otros paquetes útiles
\usepackage{enumitem}
\usepackage{microtype}
\usepackage{fancyhdr}
\usepackage{titlesec}

\usepackage{ragged2e}
\usepackage{multicol}

% Para las lineas horizontales de cabecera y footer
\usepackage{xcolor} % Líneas de colores

% Encabezados y pies de página
\pagestyle{fancy}
\fancyhead{}
\fancyfoot{}
% \fancyhead[LE,RO]{\thepage} % \thepage es el número de página
\fancyhead[CE]{Ritmos Irregulares}
\fancyhead[CO]{\nouppercase{\rightmark}}
% \fancyhead[C]{\nouppercase{\rightmark}}
% \fancyfoot[L]{Ritmos Irregulares} % Pie de página (izq)
% \fancyfoot[R]{\thepage} % Pie de página (der)

% \renewcommand{\headrulewidth}{0pt} % Quita la línea de la cabecera

% Línea horizontal en la cabecera
\renewcommand{\headrulewidth}{0.4pt}
\renewcommand{\headrule}{%
  {\color{gray!60}
    \hrule width\headwidth height0.4pt
  }
}

% Línea discontinua en la cabecera
% \usepackage{dashrule}
% \renewcommand{\headrule}{
%   {\color{gray!60}
%     \hdashrule[0.5ex]{\headwidth}{0.4pt}{2mm 1.5mm}
%   }
% }


\fancyfoot[LE,RO]{\thepage} % Número de página en exterior
\fancyfoot[CE]{Santiago Chávez Novaro} % Autor en centro exterior
\fancyfoot[CO]{Judith de León} % Autor en centro interior
% \fancyfoot[C]{Ritmos Irregulares} % Título centrado
% \fancyfoot[LO,RE]{Judith de León} % Autor en interior (opcional)

% \renewcommand{\footrulewidth}{0.4pt}

% Línea horizontal gris en el pie de página
\renewcommand{\footrulewidth}{0.4pt}
\renewcommand{\footrule}{%
  {\color{gray!60}
    \hrule width\headwidth height0.4pt
  }
}

\usepackage[font=bf]{caption} % 'font=bf' makes the caption bold, remove if not needed
\captionsetup[figure]{labelformat=empty,labelsep=none}

% Espaciado entre líneas
\usepackage{setspace}
\onehalfspacing

% Capítulos y secciones más vistosos
\titleformat{\chapter}[display]
  {\normalfont\huge\bfseries}{\chaptername\ \thechapter}{20pt}{\Huge}

% Imágenes en su sitio
\usepackage[section]{placeins}

% Bibliografía (opcional)
% \usepackage[backend=biber,style=apa]{biblatex}
% \addbibresource{bibliografia.bib}

% Si usas XeLaTeX o LuaLaTeX y quieres fuentes Unicode:
\usepackage{fontspec}
\setmainfont{Latin Modern Roman}

% Para evitar que la última página antes de cada \pagebreak (o salto de página) extienda el contenido verticalmente y deje un espacio vacío después del último párrafo
\raggedbottom

% Fin del preámbulo